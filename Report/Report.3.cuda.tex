\documentclass[letterpaper,12pt]{article}
\usepackage{tabularx}
\usepackage{amsmath}  
\usepackage{graphicx}
\usepackage[margin=1in,letterpaper]{geometry} 
\usepackage{cite}
\usepackage[final]{hyperref}
\usepackage{listings}

\lstdefinestyle{CStyle}{
    backgroundcolor=\color{backgroundColour},   
    commentstyle=\color{mGreen},
    keywordstyle=\color{magenta},
    numberstyle=\tiny\color{mGray},
    stringstyle=\color{mPurple},
    basicstyle=\footnotesize,
    breakatwhitespace=false,         
    breaklines=true,                 
    captionpos=b,                    
    keepspaces=true,                 
    numbers=left,                    
    numbersep=5pt,                  
    showspaces=false,                
    showstringspaces=false,
    showtabs=false,                  
    tabsize=2,
    language=C
}

\begin{document}

\title{Report 1}
\author{Agwu Chinedu}
\date{28th August 2018}
\maketitle


\section{Task}
Explain how you implement the labwork \\
What’s the speedup?\\
Try experimenting with different block size values\\


\section{Result}
1. Host feeds device with data\\
2. Host asks device to process data\\
3. Device processes data in parallel\\
4. Device returns result\\
\paragraph{}
The lapsed time is 133.4ms making it much faster than CPU computing time.
\paragraph{}
The lager the block size the fewer the number of blocks we have. There by enhancing the throughput of the GPU.




\end{document}